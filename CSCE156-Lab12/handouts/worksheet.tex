\documentclass[12pt]{scrartcl}

\usepackage{fullpage}

\setlength{\parindent}{0pt}
\setlength{\parskip}{.25cm}

\usepackage{graphicx}

\usepackage{xcolor}

\definecolor{darkred}{rgb}{0.5,0,0}
\definecolor{darkgreen}{rgb}{0,0.5,0}
\usepackage{hyperref}
\hypersetup{
  letterpaper,
  colorlinks,
  linkcolor=red,
  citecolor=darkgreen,
  menucolor=darkred,
  urlcolor=blue,
  pdfpagemode=none,
  pdftitle={CS2 - Lab Worksheet},
  pdfkeywords={}
}

\definecolor{MyDarkBlue}{rgb}{0,0.08,0.45}
\definecolor{MyDarkRed}{rgb}{0.45,0.08,0}
\definecolor{MyDarkGreen}{rgb}{0.08,0.45,0.08}

\definecolor{mintedBackground}{rgb}{0.95,0.95,0.95}
\definecolor{mintedInlineBackground}{rgb}{.90,.90,1}

%\usepackage{newfloat}
\usepackage[newfloat=true]{minted}
\setminted{mathescape,
               linenos,
               autogobble,
               frame=none,
               framesep=2mm,
               framerule=0.4pt,
               %label=foo,
               xleftmargin=2em,
               xrightmargin=0em,
               startinline=true,  %PHP only, allow it to omit the PHP Tags *** with this option, variables using dollar sign in comments are treated as latex math
               numbersep=10pt, %gap between line numbers and start of line
               style=default, %syntax highlighting style, default is "default"
               			    %gallery: http://help.farbox.com/pygments.html
			    	    %list available: pygmentize -L styles
               bgcolor=mintedBackground} %prevents breaking across pages
               
\setmintedinline{bgcolor={mintedBackground}}
\setminted[text]{bgcolor={mintedBackground},linenos=false,autogobble,xleftmargin=1em}
%\setminted[php]{bgcolor=mintedBackgroundPHP} %startinline=True}
\SetupFloatingEnvironment{listing}{name=Code Sample}
\SetupFloatingEnvironment{listing}{listname=List of Code Samples}

\begin{document}

\section*{CSCE 156 - Lab 12.0 - Recursion - Worksheet}

Names: \underline{\hspace{10cm}}


\begin{enumerate}
  \item Activity 1: Modify the Fibonacci code as specified 
  and answer the following questions
  \begin{enumerate}
    \item When computing \mintinline{java}{fibonacci(10)}, how 
    many times would fibonacci(5) be called?
    \item When computing \mintinline{java}{fibonacci(20)}, how 
    many times would \mintinline{java}{fibonacci(10)} be called?
    \item How long does it take for \mintinline{java}{fibonacci(45)} 
    to execute?
    \item Give an estimate of an asymptotic characterization 
    of the number of times the function is called when 
    \mintinline{java}{fibonacci(n)} is computed: is it 
    constant, linear, quadratic, cubic, or exponential?
  \end{enumerate}
  
  \item Activity 3: Modify the code that renders the Sierpinski 
  Triangle as described in Case Study 3 and answer the following 
  questions.
  \begin{enumerate}
    \item For a depth of 4 (\mintinline{java}{recursions = 4}), 
    how many triangles are drawn?
    \item For a depth of 10, how many triangles are drawn?    
    \item For a depth of 13, how many triangles are drawn??    \item (Optional) Without actually running it (the application 
    will most likely crash), can you determine how many triangles 
    are drawn for a depth of 20?
  \end{enumerate}
  
  \item Activity 4: Demonstrate your working program, what are 
  the first 5 and last 5 digits of the 1000-th Pell Number?
\end{enumerate}

Lab Instructor Signature\underline{\hspace{7.5cm}}

\end{document}
